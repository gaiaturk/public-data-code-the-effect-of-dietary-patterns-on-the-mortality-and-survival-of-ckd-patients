%\thispagestyle{plain}
\section{References}
\begin{flushleft}

[1]	The National Institute of Diabetes and Digestive and Kidney Diseases. What Is Chronic 
Kidney Disease? https://www.niddk.nih.gov/health-information/kidney-disease/chronic 
-kidney-disease-ckd/what-is-chronic-kidney-disease

[2]	Jaimon T. K., Suetonia C. P., Shu N. W., Marinella R., Juan-Jesus C., Katrina L. C., and Giovanni F. M. S.  Healthy Dietary Patterns and Risk of Mortality and ESRD in CKD: A Meta-Analysis of Cohort Studies 

[3]	Chen X, Wei G, Jalili T, Metos J, Giri A, Cho ME, Boucher R, Greene T, Beddhu S: The associations of plant protein intake with all-cause mortality in CKD. Am J Kidney Dis 67: 423–430, 2016 (26)

[4]	Gutierrez O.M., Muntner P, Rizk D.V., McClellan W.M., Warnock D.G., Newby P.K., Judd S.E.: Dietary patterns and risk of death and progression to ESRD in individuals with CKD: A cohort study. Am J Kidney Dis 64: 204–213, 2014  (27)

[5]	Huang X., Jime nez-Moleo J. J., Lindholm B., Cederholm T., Arnoldv J., Rise rus U., Sjogren P., Carrero J. J.: Mediterranean diet, kidney function, and mortality in men with CKD. Clin J Am Soc Nephrol 8: 1548–1555, 2013 (28)

[6]	Muntner P, Judd S.E., Gao L, Gutierrez O.M., Rizk D.V., McClellanW., Cushman M., Warnock D.G.: Cardiovascular risk factors in CKD associated with both ESRD and mortality. J Am Soc Nephrol 24: 1159–1165, 2013 (29)

[7]	Ricardo A.C., Madero M., Yang W., Anderson C., Menezes M., Fischer M.J., Turyk M., Daviglus M.L., Lash J.P.: Adherence to a healthy lifestyle and all-cause mortality in CKD. Clin J Am Soc Nephrol 8: 602–609, 2013 (30)

[8]	Tsuruya K., Fukuma S., Wakita T., Ninomiya T., Nagata M., Yoshida H., Fujimi S., Kiyohara Y., Kitazono T., Uchida K., Shirota T., Akizawa T., Akiba T., Saito A., Fukuhara S.: Dietary patterns and clinical outcomes in hemodialysis patients in Japan: A cohort study. PLoS One 10: e0116677, 2015 (31)

[9]	Ricardo A.C., Anderson C.A., Yang W., Zhang X., Fischer M. J., Dember L. M., Fink J. C., Frydrych A., Jensvold N. G., Lustigova E., Nessel L. C., Porter A. C., Rahman M., Wright Nunes J. A., Daviglus M. L., Lash J. P.; CRIC Study Investigators: Healthy lifestyle and risk of kidney disease progression, atherosclerotic events, and death in CKD: Findings from the Chronic Renal Insufficiency Cohort (CRIC) Study. Am J Kidney Dis 65: 412–424, 2015 (17)

\noindent [10]	Centers for Disease Control and Prevention. National Health and Nutrition Examination Survey. https://wwwn.cdc.gov/nchs/nhanes/search/datapage.aspx?Component=Dietary. 

[11]	Health.gov. Shifts Needed to Align With Healthy Eating Patterns. https://health.gov/
dietaryguidelines/2015/guidelines/chapter-2/a-closer-look-at-current-intakes-and-recommended-shifts/

[12]	Agricultural Research Service (ARS), USDA. Food Code Numbers and the Food Coding 
Scheme . https://reedir.arsnet.usda.gov/codesearchwebapp/\(gcp3kq55ssdyc445ry2k2rus\)/
coding\_scheme.pdf 

[13]	The U.S. Department of Agriculture’s (USDA) . Vegetable Subgroups. http://www.cn.nysed.gov/common/cn/files/Vegetable\%20Subgroups.pdf 

[14]	Centers for Disease Control and Prevention. Key Concepts About the USDA Food 
Coding Scheme. https://www.cdc.gov/nchs/tutorials/Dietary/SurveyOrientation/
ResourceDietaryAnalysis/Info2.htm 

[15]	Health.gov. Appendix 3. USDA Food Patterns: Healthy U.S.-Style Eating Pattern. https://health.gov/dietaryguidelines/2015/guidelines/appendix-3/

[16]	United states Renal Data System (USRDS). The 2018 USRDS Annual Data Report Reference Tables: https://www.usrds.org/reference.aspx

[17]	United states Renal Data System (USRDS). 2018 ADR Chapters. https://www.usrds.org/
2018/view/Default.aspx

[18]	ARS, USDA. Food Surveys Research Group: Beltsville, MD . Documentation and 
Dataset: https://www.ars.usda.gov/northeast-area/beltsville-md-bhnrc/beltsville-human-
nutrition-research-center/food-surveys-research-group/docs/wweia-documentation-and-data-setsY

[19]	United states Renal Data System (USRDS). 2018 USRDS Annual Data Report: 
Executive Summary. https://www.usrds.org/2018/download/v1\_00\_ExecSummary\_18.pdf

[20]	Health.gov, CDC. Dietary  Guidelines  for Americans  2015-2020. https://health.gov/dietaryguidelines/2015/resources/2015-2020\_dietary\_guidelines.pdf 

[21]	Health.gov, CDC. Dietary  Guidelines  for Americans. https://health.gov/dietaryguidelines/dga95/9DIETGUI.HTM

[22]	Mortality and Causes of Death. https://www.usrds.org/2018/ref/ESRD\_Ref\_H\_Mortality
\_2018.xlsx

[23]	 Tong A, Chando S, Crowe S, Manns B, Winkelmayer WC, Hemmelgarn B, Craig JC: Research priority setting in kidney disease: A systematic review. Am J Kidney Dis 65: 674–683, 2015

[24]	Lin J, Fung TT, Hu FB, Curhan GC: Association of dietary patterns with albuminuria and kidney function decline in older white women: A subgroup analysis from the Nurses’ Health Study. Am J Kidney Dis 57: 245–254, 2011 

[25]	Taylor EN, Fung TT, Curhan GC: DASH-style diet associates with reduced risk for kidney stones. J Am Soc Nephrol 20: 2253–2259, 2009 

[26]	Liu, Hao-Wen; Tsai, Wen-Hsin; Liu, Jia-Sin; Kuo, Ko-Lin. 2019. "Association of Vegetarian Diet with Chronic Kidney Disease." Nutrients 11, no. 2: 279.

[27]	Golaleh Asghari, Mehrnaz Momenan, Emad Yuzbashian, Parvin MirmiranEmail author and Fereidoun Azizi. Dietary pattern and incidence of chronic kidney disease among adults: a population-based study

[28]	Tanushree Banerjee1 , Deidra C. Crews2 , Delphine S. Tuot3 , Meda E. Pavkov4 , Nilka Rios Burrows4 , Austin G. Stack5 , Rajiv Saran6,7 , Jennifer Bragg-Gresham6 and Neil R. Powe1,8 ; for the Centers for Disease Control and Prevention Chronic Kidney Disease Surveillance Team9 Poor accordance to a DASH dietary pattern is associated with higher risk of ESRD among adults with moderate chronic kidney disease and hypertension

[29]	 Jacek R., Beata F., Aleksandra C., Anna G.The Effect of Diet on the Survival of Patients with Chronic Kidney Disease. Nutrients 2017, 9(5), 495; https://doi.org/10.3390/nu9050495

[30] National Kidney Foundation. One in Seven American Adults Estimated to Have Chronic Kidney Disease. https://www.kidney.org/news/one-seven-american-adults-estimated-to-have-chronic-kidney-disease

[31]	Hannah N. What are the leading causes of death in the US?. https://www.medical 
newstoday.com/articles/282929.php 

[32]	Davita. Five Stages of CKD. https://www.davita.com/education/kidney-disease/stages

[33]	Goal: How to Identify the Most Important Predictor Variables in Regression Models https 
://blog. minitab.com/blog/adventures-in-statistics-2/how-to-identify-the-most-important- predictor-variables-in-regression-models

[34]	How to Interpret Regression Analysis Results: P-values and Coefficients https://blog. 
minitab.com/blog/adventures-in-statistics-2/how-to-interpret-regression-analysis-results-p-
values-and-coefficients

[35]	Regression Analysis: How to Interpret the Constant (Y Intercept) https://blog.minitab
.com/blog/adventures-in-statistics-2/regression-analysis-how-to-interpret-the-constant-y
-intercept

[36]	How to Compare Regression Slopes: How to statistically test the difference between 
regression slopes and constants https://blog.minitab.com/blog/adventures-in-statistics-2/how
-to-compare-regression-lines-between-different-models

[37]	How Do I Interpret R-squared and Assess the Goodness-of-Fit? https://blog.minitab.com
/blog/adventures-in-statistics-2/regression-analysis-how-do-i-interpret-r-squared-and-assess
-the-goodness-of-fit

[38]	How High Should R-squared Be in Regression Analysis? https://blog.minitab.com/blog/
adventures-in-statistics-2/how-high-should-r-squared-be-in-regression-analysis

[39]	How to Interpret a Regression Model with Low R-squared and Low P values https://blog
.minitab.com/blog/adventures-in-statistics-2/how-to-interpret-a-regression-model-with-low-r-
squared-and-low-p-values

[40]	Use Adjusted R-Squared and Predicted R-Squared to Include the Correct Number of 
Variables https://blog.minitab.com/blog/adventures-in-statistics-2/multiple-regession-
analysis-use-adjusted-r-squared-and-predicted-r-squared-to-include-the-correct-number-of-
variables

[41]	How to Interpret S, the Standard Error of the Regression https://blog.minitab.com/blog/
adventures-in-statistics-2/regression-analysis-how-to-interpret-s-the-standard-error-of-the 
-regression

[42]	What Is the F-test of Overall Significance in Regression Analysis?. https://blog.
minitab.com/blog/adventures-in-statistics-2/what-is-the-f-test-of-overall-significance-in-
regression-analysis

[43]	Understanding Analysis of Variance (ANOVA) and the F-test. https://blog.minitab.com/
blog/adventures-in-statistics-2/understanding-analysis-of-variance-anova-and-the-f-test

[44]	How to Compare Regression Slopes. https://blog.minitab.com/blog/adventures-in-
statistics-2/how-to-compare-regression-lines-between-different-models

[45]	How to Present and Use the Results to Avoid Costly Mistakes, part 1. https://blog.
minitab.com/blog/adventures-in-statistics-2/applied-regression-analysis-how-to-present-and-
use- the-results-to-avoid-costly-mistakes-part-1

[46]	How to Identify the Most Important Predictor Variables in Regression Models. https://
blog.minitab.com/blog/adventures-in-statistics-2/how-to-identify-the-most-important-
predictor- variables-in-regression-models

[47]	How to Interpret your Regression Results http://sitestree.com/how-to-interpret-your- 
regression-results

[48]  Fernandez-Prado R., Esteras R., Perez-Gomez M.V, Gracia-Iguacel C., Gonzalez-Parra E.,  Sanz A.B., Ortiz A., Sanchez-Nino M.D. Nutrients Turned into Toxins: Microbiota Modulation of Nutrient Properties in Chronic Kidney Disease. Nutrients. 2017 May; 9(5): 489. Published online 2017 May 12. doi: 10.3390/nu9050489 PMCID: PMC5452219 https://www.ncbi.nlm.nih.gov/pmc/articles/PMC5452219/

[49]  Hsu, Y. H., Pai, H. C., Chang, Y. M., Liu, W. H.,  Hsu, C. C. (2013). Alcohol consumption is inversely associated with stage 3 chronic kidney disease in middle-aged Taiwanese men. BMC nephrology, 14, 254. doi:10.1186/1471-2369-14-254

[50]  Rippe, J. M.,  Angelopoulos, T. J. (2016). Relationship between Added Sugars Consumption and Chronic Disease Risk Factors: Current Understanding. Nutrients, 8(11), 697. doi:10.3390/nu8110697

[51]  Karalius V.P., Shoham D.A. Dietary sugar and artificial sweetener intake and chronic kidney disease: a review. Adv Chronic Kidney Dis. 2013; 20:157–164. doi: 10.1053/j.ackd.2012.12.005

[52] Jaimon T. Kelly, Suetonia C. Palmer, Shu Ning Wai, Marinella Ruospo, Juan-Jesus Carrero, Katrina L. Campbell and Giovanni F. M. Strippoli Healthy Dietary Patterns and Risk of Mortality and ESRD in CKD: A Meta-Analysis of Cohort Studies. CJASN February 2017, 12 (2) 272-279; DOI: https://doi.org/10.2215/CJN.06190616 

[53]  Jilian k. The 20 Best Foods for People With Kidney Problems. https://www.healthline.
com/nutrition/best-foods-for-kidneys

[54]  National Kidney Foundation (NKF), USA. Drinking Alcohol Affects Your Kidneys https://www.kidney.org/news/kidneyCare/winter10/AlcoholAffects

[55]  Uehara, S., Hayashi, T., Kogawa Sato, K., Kinuhata, S., Shibata, M., Oue, K., Hashimoto, K. (2016). Relationship Between Alcohol Drinking Pattern and Risk of Proteinuria: The Kansai Healthcare Study. Journal of epidemiology, 26(9), 464–470. doi:10.2188/jea.JE20150158

[56]  Nettleton, J. A., Steffen, L. M., Palmas, W., Burke, G. L., \& Jacobs, D. R., Jr (2008). Associations between micro albuminuria and animal foods, plant foods, and dietary patterns in the Multiethnic Study of Atherosclerosis. The American journal of clinical nutrition, 87(6), 1825–1836. doi:10.1093/ajcn/87.6.1825

[57] 	Jacobs, D. R., Jr, Gross, M. D., Steffen, L., Steffes, M. W., Yu, X., Svetkey, L. P.,  Sacks, F. (2009). The effects of dietary patterns on urinary albumin excretion: results of the Dietary Approaches to Stop Hypertension (DASH) Trial. American journal of kidney diseases : the official journal of the National Kidney Foundation, 53(4), 638–646. doi:10.1053/j.ajkd.2008.10. 048. https://www.ncbi.nlm.nih.gov/pmc/articles/PMC2676223/

[58] National Kidney Foundation. 37 Million American Adults Now Estimated to Have Chronic Kidney Disease. https://www.kidney.org/news/37-million-american-adults-now-estimated-to-have-chronic-kidney-disease
\end{flushleft}