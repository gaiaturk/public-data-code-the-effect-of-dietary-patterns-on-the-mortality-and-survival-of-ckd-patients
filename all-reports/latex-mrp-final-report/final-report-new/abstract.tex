\begin{center} 
\textbf{Effect of Dietary Patterns on Chronic Kidney Disease (CKD) Measures (ACR), and on the Mortality of CKD Patients} 

Sayed Ahmed \\
Master of Science 2019 \\
Data Science and Analytics \\
Ryerson University \\

\end{center}
%\thispagestyle{plain}
\begin{center} \section*{Abstract} \end{center}
Chronic Kidney Disease (CKD) leading to End-Stage Renal Disease (ESRD) is very prevalent today. Over 37 millions of Americans have CKD. CKD/ESRD and interrelated diseases cause a majority of the early deaths.  Many researchers studied the effect of drugs on CKD. Little research studied the effect of  whole dietary patterns. This research has identified the effect of dietary patterns on CKD mortality; also on a CKD measure named Albumin Creatinine Ratio (ACR). Dietary surveys from NHANES, and CKD Mortality dataset from USRDS were utilized. Principal Component Analysis and Regression were utilized to find the effect. Machine Learning Approaches including Regression, and Bayesian were applied to predict ACR values. Grains, Other Vegetables showed positive correlations with Mortality where Alcohol, Sugar, and Nuts showed negative correlations. ACR values were not found strongly correlated with dietary patterns. For ACR value prediction, 10 Fold Cross Validations with Polynomial Regression showed 95\% accuracy.

\medskip
\noindent \textbf{Keywords:} 

\noindent CKD, ESRD, Dietary Patterns, Mortality, ACR, Polynomial Regression, Bayesian